\documentclass[a4paper,12pt]{article}

\usepackage[utf8]{inputenc} % Encoding no formato UTF-8
\usepackage[brazil]{babel}  % Pacote para a língua portuguesa

\usepackage{setspace}               % espaçamento flexível
\usepackage{indentfirst}            % indentação do primeiro parágrafo
\usepackage[fixlanguage]{babelbib}  
\usepackage[usenames,svgnames,dvipsnames]{xcolor}

\usepackage[pdftex,plainpages=false,pdfpagelabels,pagebackref,
            colorlinks=true,citecolor=DarkGreen,linkcolor=DarkRed,
            urlcolor=DarkRed,filecolor=DarkGreen,
            bookmarksopen=true]{hyperref}
            % links coloridos
\usepackage[all]{hypcap} % soluciona o problema com hiperreferências 

% -------------------------------------------------------------------- %
% INFORMAÇÕES
\pdfinfo{
  /title    (Localização de Palavras II)
  /author   (Renato Cordeiro Ferreira)
  /creator  (Renato Cordeiro Ferreira)
  /producer (Renato Cordeiro Ferreira)
  /subject  (Proposta de Iniciação Científica)
  /keywords (Lematização, hashes, probing linear, tabelas de símbolos)
}

\title {Localização de Palavras II}
\author{Renato Cordeiro ferreira}

\begin{document}

\newpage %%%%%%%%%%%%%%%%%%%%%%%%%%%%%%%%%%%%%%%%%%%%%%%%%%%%%%%%%%%%%%%
\pagenumbering{arabic} % começamos a numerar 

\maketitle

\section{Introdução} 
%%%%%%%%%%%%%%%%%%%%%%%%%%%%%%%%%%%%%%%%%%%%%%%%%%%%%%%%%%%%%%%%%%%%%%%%

    Neste exercício-programa, tínhamos como objetivo criar uma estrutura
    que manipulasse as saídas geradas pelo programa CoreNLP,     
    disponibilizado pela Universidade de Stanford\footnote{Disponível em
    \url{http://nlp.stanford.edu/software/corenlp.shtml}}. O CoreNLP é 
    um software de processamento de linguagem natural que oferece o 
    recurso de ``lematização'': para cada palavra presente num texto,
    geramos o correspondente ``lema'' - a origem da palavra.

    Para este programa, foram utilizadas as estruturas de tabelas de
    hash com solução de conflitos por \textit{probing linear} e por
    encadeamento, de modo a obter algoritmos de complexidade constante
    (O(1)) para inserções e remoções na maioria dos casos.
    
\section{Arquivos} 
%%%%%%%%%%%%%%%%%%%%%%%%%%%%%%%%%%%%%%%%%%%%%%%%%%%%%%%%%%%%%%%%%%%%%%%%
    
    O presente EP é composto dos seguintes arquivos:    
    
    \begin{itemize}
      \item \href{run:./Item.h}{Item.h}
      \item \href{run:./list.c}{list.c}
      \item \href{run:./list.h}{list.h}
      \item \href{run:./arne.c}{arne.c}
      \item \href{run:./arne.h}{arne.h}
      \item \href{run:./enc.c}{enc.c}
      \item \href{run:./enc.h}{enc.h}
      \item \href{run:./lp.c}{lp.c}
      \item \href{run:./lp.h}{lp.h}
      \item \href{run:./word.c}{word.c}
      \item \href{run:./word.h}{word.h}
      \item \href{run:./lemma.c}{lemma.c}
      \item \href{run:./lemma.h}{lemma.h}
      \item \href{run:./getline.c}{getline.c}
      \item \href{run:./getline.h}{getline.h}
      \item \href{run:./main.c}{main.c}
      \item \href{run:./Makefile}{Makefile}
    \end{itemize}
    
    Cada um deles apresenta algumas particularidades:
    
    \subsection{item} 
    %%%%%%%%%%%%%%%%%%%%%%%%%%%%%%%%%%%%%%%%%%%%%%%%%%%%%%%%%%%%%%%%%%%%
        O item generalizado que serve como configuração para as     
        implementações das tabelas de símbolos. Neste EP, fizemos uma 
        modicação em relação ao anterior, transformando as chaves de
        tipos \textbf{void *} para \textbf{char *}.

    \subsection{list} 
    %%%%%%%%%%%%%%%%%%%%%%%%%%%%%%%%%%%%%%%%%%%%%%%%%%%%%%%%%%%%%%%%%%%%
        Uma lista generalizada, implementada com o auxílio de ponteiros
        do tipo \textbf{void *} e com tipos de 1ª classe. É possível
        usá-la como uma lista genérica para itens de qualquer tipo.

    \subsection{arne} 
    %%%%%%%%%%%%%%%%%%%%%%%%%%%%%%%%%%%%%%%%%%%%%%%%%%%%%%%%%%%%%%%%%%%%
        Tabela de símbolos implementada com o uso de \textbf{ARNEs}
        (Árvores Rubro-Negro Esquerdistas). Estas árvores foram
        generalizadas para a utilização como tipo de 1ª classe e para
        aceitarem qualquer tipo de estrutura (por meio de ponteiros 
        \textbf{void *}).

        As árvores rubro-negras foram baseadas no código disponibilizado
        na página do professor Yoshiharu
        (\href{http://www.ime.usp.br/~yoshi/2012i/mac323/exx/LLRB.3/llrb.c}{llrb.c})
        e modificados para aceitar diferentes tipos de chaves. A função de 
        definição da chave precisa ser especificada juntamente com as
        funções ``less'', ``eq'' e o tipo ``NULLitem'' na função 
        ``STinit()``.
        
        Para este EP, os arquivos das ARNEs foram mantidas. Entretanto,
        suas implementações não são utilizadas.

    \subsection{enc} 
    %%%%%%%%%%%%%%%%%%%%%%%%%%%%%%%%%%%%%%%%%%%%%%%%%%%%%%%%%%%%%%%%%%%%
        O arquivo correspondente a enc.c implementa um protótipo de
        tabela de símbolo com várias funções de hash baseadas na
        implementação do Prof. Yoshiharu e generalizadas como tipos de
        1ª classe para poder ser usado entre as tabelas T1 e T2. A
        solução das colisões se dá por meio de encadeamentos (criação de
        listas ligadas) que auxiliam a colocar itens de mesma chave. O
        hash usado foi simples e a tabela se redimensiona sempre que a
        proporção N/M > 1/2 (N o total de objetos, M o total de espaços
        na tabela).
    \subsection{pl} 
    %%%%%%%%%%%%%%%%%%%%%%%%%%%%%%%%%%%%%%%%%%%%%%%%%%%%%%%%%%%%%%%%%%%%
        O arquivo correspondente a pl.c implementa um protótipo de
        tabela de símbolo com várias funções de hash baseadas na
        implementação do Prof. Yoshiharu e generalizadas como tipos de
        1ª classe para poder ser usado entre as tabelas T1 e T2. A
        solução das colisões se dá por meio de probing linear
        (saltos dentro do vetor de chaves) que auxilia a colocar itens
        de mesma chave distribuídos no vetor. O
        hash usado foi simples e a tabela se redimensiona sempre que a
        proporção N/M > 1/2 (N o total de objetos, M o total de espaços
        na tabela).

    \subsection{word} 
    %%%%%%%%%%%%%%%%%%%%%%%%%%%%%%%%%%%%%%%%%%%%%%%%%%%%%%%%%%%%%%%%%%%%
        Módulo relacionado a funções de manipulação da tabela de
        símbolos que armazena palavras. Contém estruturas que guardam
        uma lista de sentenças, o lema e a própria palavra (por meio de 
        ponteiros para um buffer) que auxiliam a imprimir, contar e 
        acessar os dados presentes na tabela de símbolos.

    \subsection{lemma} 
    %%%%%%%%%%%%%%%%%%%%%%%%%%%%%%%%%%%%%%%%%%%%%%%%%%%%%%%%%%%%%%%%%%%%

        Um módulo que utiliza as estruturas de lista e tabela de
        símbolos para poder construir as funções necessárias para o
        acesso, contagem e impressão de dados relacionados aos lemas.

    \subsection{getline}
    %%%%%%%%%%%%%%%%%%%%%%%%%%%%%%%%%%%%%%%%%%%%%%%%%%%%%%%%%%%%%%%%%%%%
        
        O arquivo 'getline', retirado das notas de aula do prof.
        Yoshiharu, foi utilizado como um módulo para leitura de strings
        de tamanho desconhecido. Com ele, foi possível produzir um
        buffer que armazenasse o texto lido.

    \subsection{main} 
    %%%%%%%%%%%%%%%%%%%%%%%%%%%%%%%%%%%%%%%%%%%%%%%%%%%%%%%%%%%%%%%%%%%
        Arquivo principal do texto que contêm a interface com o usuário
        (via linha de comando e opções de execução) e a leitura (em fase
        de pré-processamento) do texto passado como parâmetro ao
        programa.

    \subsection{makefile} 
    %%%%%%%%%%%%%%%%%%%%%%%%%%%%%%%%%%%%%%%%%%%%%%%%%%%%%%%%%%%%%%%%%%%%
        Arquivo para a compilação dos arquivos. Sua estrutura é de
        propósito geral e permite compilar projetos diversificados com
        diretórios.
        
\section{Considerações finais} 
%%%%%%%%%%%%%%%%%%%%%%%%%%%%%%%%%%%%%%%%%%%%%%%%%%%%%%%%%%%%%%%%%%%%%%%%
    Dada a generalização das estruturas propostas por esta implementação
    do exercício-programa, o custo do uso de ponteiros \textbf{void *}
    durante quase todo o processo diminuiu a eficiência do
    pré-processamento e carregamento das estruturas de dados.

    Isso não impediu, porém, que o acesso e impressão dos dados fossem
    realizados de forma eficiênte, dado que os algoritmos de
    complexidade logarítmica (O(n)) inerentes às ARNEs mantiveram-se
    assimptoticamente ótimos para a estrutura.

\end{document}
